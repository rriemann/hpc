\section*{Aufgabe 1}
\subsection*{Schleife 1}

Die erste Schleife kann nicht parallelisiert werden, da in ihr \texttt{break}
aufgerufen wird, was nicht parallelisierbar ist (s. Vorlesung). Wenn man sie parallelisieren
würde, wüssten die einzelnen Threads nicht, wann in einem der Threads die Abbruchbedingung 
erfüllt ist und würden weiterlaufen, ohne dass dies gewünscht ist. Die Funktionalität
wäre also nicht die gleiche wie bei einem Single-Thread-Aufruf.

\subsection*{Schleife 2}

Die zweite Schleife hingegen lässt sich parallelisieren. Hier wird nur aus einem
Array gelesen und schließlich, je nach Erfüllung der Bedingung, eine als global
anzulegende Variable (\texttt{foundit}) auf den Wert 1 gesetzt. Dies ist völlig
unabhängig von der Reihenfolge, in der das Array abgearbeitet wird, sodass hier
die Parallelisierung mit \texttt{OpenMP} möglich ist.

\subsection*{Schleife 3}

Diese Schleife ist ebenfalls parallelisierbar, da in der dargestellten Form ausschließlich
aus zwei verschiedenen Arrays gelesen wird und als Linearkombination in ein weiteres
Array geschrieben werden. Es gibt also auch hier keine Abhängigkeiten zwischen
den in den verschiedenen Threads abgearbeiteten Arrays oder zwischen zwischen Elementen
eines Arrays.

\subsection*{Schleife 4}

Die letzte zu betrachtende Code-Einheit besteht aus zwei ineinander verschachtelten
\texttt{for}-Schleifen. Es ist zu untersuchen, welche der beiden Schleifen parallelisiert
werden kann. Da man mit \texttt{OpenMP} nur jeweils eine der beiden Schleifen parallelisieren
kann, muss man sich keine Gedanken darüber machen, ob es zu Konflikten kommen könnte,
wenn man dies versuchen würde.

Die äußere Schleife ließe sich parallelisieren. Alles, was in ihr ausgeführt wird,
hängt direkt von ihrer Laufvariable $i$ ab (\texttt{size[i]} und \texttt{a[i][j+1]}).
Daher gibt es hier, wie schon im Falle der dritten Schleife argumentiert wurde, keine
Thread-Grenzen-Konflikte oder Abhängigkeiten des mit $i$ indizierten Arrays.

Die innere Schleife kann, wenn $i$ als globale Variable deklariert wird, ebenfalls
parallelisiert werden. Es besteht zwar eine Abhängigkeit zwischen den Elementen
von $a$, aber diese wirkt nur lesend auf die Nachfolgenden Elemente, also insbesondere
nicht auf vorhergehende Elemente. Wäre dies der Fall, so wäre die Schleife nicht
ohne Weiteres parallelisierbar. Im vorliegenden Fall aber wird ein Element geschrieben
und im weiteren Verlauf nicht mehr verwendet. Daher ist diese Schleife parallelisierbar.

\section*{Aufgabe 2}

In dieser Aufgabe war der gegebene Code so umzuformulieren, dass man die Schleife
mit Hilfe von der \texttt{OpenMP}-Anweisung \texttt{parallel for} parallelisiert
werden könnte.

Der Code dafür könnte beispielsweise wie in \lref{code2} aussehen:

\begin{lstlisting}[caption=Parallelisierte Schleife,label=lst:code2]
y[1] = 2;
#pragma omp parallel for
for (i = 2; i <= N; i++) {
    y[i] = 2*i - i;
}
\end{lstlisting}

\section*{Aufgabe 3}

In der letzten Aufgabe war wieder ein gegebener Code zu parallelisieren, wobei dieses 
mal eine Abhängigkeit der Arrayelemente vorhandene ist. Für die Berechnung des Elements
$i$ muss das Element $i-1$ bekannt sein, sodass eine einfache Parallelisierung nicht
ohne Weiteres möglich ist.

Möglichkeit 1:
\begin{lstlisting}[caption=Parallelisierte Schleife,label=lst:code3_1]
a2 = a;				//vllt a global machen
#pragma omp parallel for
for (i = 1; i <= N; i++) {
  for (j = 0; j <= i; j++) {
    a2[i] += a[j];
  }
}
\end{lstlisting}
ist sicher nicht das performanteste...

mehr fällt mir aber spontan nicht ein... 