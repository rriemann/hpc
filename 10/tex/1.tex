\section*{Aufgabe 1}
\subsection*{Schleife 1}

Die erste Schleife kann nicht parallelisiert werden, da in ihr \texttt{break}
aufgerufen wird, was nicht parallelisierbar ist (s. Vorlesung). Wenn man sie parallelisieren
würde, wüssten die einzelnen Threads nicht, wann in einem der Threads die Abbruchbedingung 
erfüllt ist und würden weiterlaufen, ohne dass dies gewünscht ist. Die Funktionalität
wäre also nicht die gleiche wie bei einem Single-Thread-Aufruf.

\subsection*{Schleife 2}

Die zweite Schleife hingegen lässt sich parallelisieren. Hier wird nur aus einem
Array gelesen und schließlich, je nach Erfüllung der Bedingung, eine als global
anzulegende Variable (\texttt{foundit}) auf den Wert 1 gesetzt. Dies ist völlig
unabhängig von der Reihenfolge, in der das Array abgearbeitet wird, sodass hier
die Parallelisierung mit \texttt{OMP} möglich ist.