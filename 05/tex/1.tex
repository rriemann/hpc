\section*{Aufgabe 1}
In dieser Aufgabe war das Programm \texttt{daxpy} aus dem ersten Projekt zu verwenden,
um zu testen, welchen Einfluss die Zahl der insgesamt auf einem Knoten gestarteten
Prozesse auf die Rechenleistung hat. Die für das Programm einzustellenden Parameter
sind der Aufgabenstellung zu entnehmen. Der Quelltext für das Steuerungsprogramm
ist in \lref{batch} und für das \texttt{daxpy}-Programm in \lref{daxpy} dargestellt.
Die übrigen Dateien sind im Vergleich zum ersten Projekt nicht verändert worden
und daher nicht erneut gezeigt.

\lstinputlisting[label=lst:batch,caption={batch.rb}]{../code/01/batch.rb}
\lstinputlisting[label=lst:daxpy,caption={daxpy.c}]{../code/01/daxpy.c}

In \lref{batch} werden die einzustellenden Parameter festgelegt, abgesehen von
der Zahl der Knoten, da der Standardwert für die Zahl der Knoten eins ist, sodass
hierfür kein Parameter übergeben werden musste.

